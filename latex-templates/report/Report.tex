%*H**********************************************************************
% FILENAME :        Report.tex
%
% DESCRIPTION :
%       Generiic report file, based on a template I found online but
%       that I can't seem to find again :(
%
% AUTHOR :    Carlos Garcia - https://github.com/cxrlos
%
%
%H*


\documentclass[12pt]{article}
\usepackage[papersize={8.5in,11in}]{geometry}

\usepackage[T1]{fontenc}
\usepackage[utf8]{inputenc}
\usepackage[spanish]{babel}
\usepackage{graphicx}
\usepackage{xcolor}

\usepackage{amsmath,amssymb,amsthm,textcomp}

\usepackage{apacite}
\usepackage[hyphens]{url}
\usepackage{amsmath,amssymb,amsthm,textcomp}
\usepackage{enumerate}
\usepackage{multicol}
\usepackage{tikz}

\usepackage{geometry}
\geometry{left=25mm,right=25mm,
bindingoffset=0mm, top=20mm,bottom=20mm}

% Set variable name for title and footers
\newcommand{\newCommandName}{}
\newcommand\fileName{REPORT\_NAME}

\linespread{1.3}
\newcommand{\linia}{\rule{\linewidth}{0.5pt}}

\makeatletter
\renewcommand{\maketitle}{
\begin{center}
\vspace{2ex}
{\huge \textsc{\@title}}
\vspace{1ex}
\\
\linia\\
\@author \hfill \@date
\vspace{4ex}
\end{center}
}
\makeatother

\usepackage{fancyhdr}
\pagestyle{fancy}
\lhead{}
\chead{}
\rhead{}
\lfoot{\fileName}
\cfoot{}
\rfoot{Page \thepage}
\renewcommand{\headrulewidth}{0pt}
\renewcommand{\footrulewidth}{0pt}

% Listing coonfig for python
\usepackage{listings}
\lstset{
    language=Python, % Change to desired lang
    basicstyle=\ttfamily\small,
    aboveskip={1.0\baselineskip},
    belowskip={1.0\baselineskip},
    columns=fixed,
    extendedchars=true,
    breaklines=true,
    tabsize=4,
    prebreak=\raisebox{0ex}[0ex][0ex]{\ensuremath{\hookleftarrow}},
    frame=lines,
    showtabs=false,
    showspaces=false,
    showstringspaces=false,
    keywordstyle=\color[rgb]{0.627,0.126,0.941},
    commentstyle=\color[rgb]{0.133,0.545,0.133},
    stringstyle=\color[rgb]{01,0,0},
    numbers=left,
    numberstyle=\small,
    stepnumber=1,
    numbersep=10pt,
    captionpos=t,
    escapeinside={\%*}{*)}
}

% -----------------------------------------------------------------

\begin{document}

\title{\fileName}

\author{Carlos García, Tecnol\'ogico de Monterrey en Santa Fe}

\date{\today}

\maketitle

\section*{Exercise 1}
\begin{lstlisting}[label={list:first},caption=Python -- Script using spaCy.]
import json
import utils

def main():
    with open('./detected_text/2-1.json') as f:
        data = json.load(f)

    tokens = utils.convert_google_to_standard(data)
    for x in tokens:
        print(x['text'])
        print(x['boundingBox'])
        print(x['words'])
        tags = [['label'], ['general'], [str(ent)]]
        x['tags'] = tags

    outfile = f('./detected_text/2-1.json')
    with open(url, 'w') as outfile:
        json.dump(tokens, outfile, indent=4)

if __name__ == "__main__":
    main()
\end{lstlisting}

Following Listing~\ref{list:first}\ldots{} \cite{Lu2020Dec}

\Urlmuskip=0mu plus 1mu\relax
\bibliographystyle{apacite}
\bibliography{mybib.bib}
\end{document}
